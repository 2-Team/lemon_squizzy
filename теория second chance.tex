\section{Second chance}

\subsection{Теория по алгоритму Second Chance.} 

Second Chance - улучшенная версия алгоритма FIFO.Данный алгоритм использовался в Multics и BSD Unix. Все страницы имеют флаг обращения (бит R), который равен либо 0, либо 1.\\
Когда страница появляется в пямати, ее бит R равен 0. Как только мы произвели чтение страницы, ее бит становится равен 1. Это означает, что страница была использована. Когда подходит очередь страницы, проверяем ее бит R. Если он равен 0, то мы выгружаем эту страницу из памяти на жесткий диск. Если же бит R был равен 1, то мы меняем его на 0 и помещяем в конец очереди. Этим мы и даем второй шанс этой странице. Если эта страница, с того момента, как мы поместили в конец очереди, и до того, как она вернулась в начала, была использована, ее бит снова становится равен 1, и цикл  повторяется, если нет, то мы выгружаем ее на жесткий диск.\\
 Приемущество этого алгоритма над FIFO в том, что мы не удаляем нужные нам страницы. Но у алгоритма Second Сhance есть свои недостатки. Если нам нужно выгрузить какую-либо страницу на жесткий диск, а бит R у всех страниц равен 1, произойдет следущее. Бит R первой страницы равен 1. Мы помещяем ее в конец очереди, устанавливая бит 0. Делаем это для всех следующих страниц. В конце мы снова придем к этой первой странице. И ее бит окажется равен 0, потому что мы только что поменяли его. То есть мы выгружаем из памяти случайную, возможно нужную нам страницу, не давая ей второго шанса. Еще один минус алгоритма Second Chance  в том, что происходит очень много перемещений страниц.\\
 В коде используются 3 освновные переменные: 
 RamSize- Размер оперативной памяти
 calls - Количество обращений к оперативной памяти
 numbers - Количество страниц\\
Так же мы используем 3 статические функции - Search, Pagefault и Pagerep. Функция Search отвечает за то, чтое сли идет обращение к странице, ее бит R становится равным 1, если же нет, оставляет равным 0. Функция Pagefault проверяет, есть ли свободное место в памяти, если есть, то страница просто просто записывается, если нет, то вызвается функция  Pagerep. Pagerep решает, дать ли второй шанс странице или выгрузить ее на жесткий диск.
 \subsection{Код алгоритма}
 
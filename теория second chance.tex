\section{Second chance}

\subsection{Теория по алгоритму Second Chance.} 

Second Chance - улучшенная версия алгоритма FIFO.Данный алгоритм использовался в Multics и BSD Unix. Все страницы имеют флаг обращения (бит R), который равен либо 0, либо 1.\\
Когда страница появляется в пямати, ее бит R равен 0. Как только мы произвели чтение страницы, ее бит становится равен 1. Это означает, что страница была использована. Когда подходит очередь страницы, проверяем ее бит R. Если он равен 0, то мы выгружаем эту страницу из памяти на жесткий диск. Если же бит R был равен 1, то мы меняем его на 0 и помещяем в конец очереди. Этим мы и даем второй шанс этой странице. Если эта страница, с того момента, как мы поместили в конец очереди, и до того, как она вернулась в начала, была использована, ее бит снова становится равен 1, и цикл  повторяется, если нет, то мы выгружаем ее на жесткий диск.\\
 Приемущество этого алгоритма над FIFO в том, что мы не удаляем нужные нам страницы. Но у алгоритма Second Сhance есть свои недостатки. Если нам нужно выгрузить какую-либо страницу на жесткий диск, а бит R у всех страниц равен 1, произойдет следущее: бит R первой страницы равен 1. Мы помещяем ее в конец очереди, устанавливая бит 0. Делаем это для всех следующих страниц. В конце мы снова придем к этой первой странице. И ее бит окажется равен 0, потому что мы только что поменяли его. То есть мы выгружаем из памяти случайную, возможно нужную нам страницу, не давая ей второго шанса. Еще один минус алгоритма Second Chance  в том, что происходит очень много перемещений страниц, что отнимает много времени.\\
 В коде используются 3 освновные переменные: \\
 1 RamSize- Размер нашей оперативной памяти\\
 2 calls - Количество обращений к оперативной памяти \\
 3 numbers - Размер виртуальной память\\
Так же мы используем 3 основных массива:\\
   1  static int[] R - массив битов R;\\
   2 static int[] M - массив битов M. Отвечает за чистые и грязные страницы.\\
   3  static int[] frame - массив из номеров страниц, которые находятся в оперативной памяти.\\
    
В программе мы используем  3 статические функции - Search, Pagefault и Pagerep. Функция Search ищет страницу в памяти, и если она там есть, то она обновляет биты R и M. Функция Pagefault проверяет, есть ли свободное место в памяти, если есть, то страница просто просто записывается, если нет, то вызвается функция  Pagerep. Pagerep выбирает стриницу для замещения. Она смотрит на бит R. Если он равен 1, то Pagerep его обнуляет, давая второй шанс. Если бит R равен 0, то выгружаем ее на диск.
 \subsection{Код алгоритма}
 